% Intended LaTeX compiler: pdflatex
\documentclass[10pt,article]{article}
\usepackage[utf8]{inputenc}
\usepackage[T1]{fontenc}
\usepackage{graphicx}
\usepackage{grffile}
\usepackage{longtable}
\usepackage{wrapfig}
\usepackage{rotating}
\usepackage[normalem]{ulem}
\usepackage{amsmath}
\usepackage{textcomp}
\usepackage{amssymb}
\usepackage{capt-of}
\usepackage{hyperref}
\usepackage{titling} \posttitle{\par\end{center}} \setlength{\droptitle}{-30pt} \usepackage{multicol} \setlength{\columnsep}{1cm} \usepackage[T1]{fontenc} \usepackage[utf8]{inputenc} \renewcommand{\contentsname}{Table of Contents / Agenda} \usepackage[letterpaper,left=1in,right=1in,top=0.7in,bottom=1in,headheight=23pt,includehead,includefoot,heightrounded]{geometry} \usepackage{fancyhdr} \pagestyle{fancy} \fancyhf{} \cfoot{\thepage} \usepackage{mathpazo} \usepackage[scaled=0.85]{helvet} \usepackage{courier} \usepackage[onehalfspacing]{setspace} \usepackage[framemethod=default]{mdframed} \usepackage{wrapfig} \usepackage{booktabs} \usepackage[outputdir=Lectures]{minted}
\setcounter{secnumdepth}{3}
\date{\vspace{-6ex}}
\title{Class 1: Regression Analysis: Introduction}
\hypersetup{
 pdfauthor={},
 pdftitle={Class 1: Regression Analysis: Introduction},
 pdfkeywords={},
 pdfsubject={Description School specific teaching materials},
 pdfcreator={Emacs 26.1 (Org mode 9.1.13)}, 
 pdflang={English}}
\begin{document}

\maketitle
\lhead{ COURSE 0000 \\ Joon H. Ro } 
\rhead{ Class 1 \\ 2018-08-28 Tue} 
\thispagestyle{fancy}

\setcounter{tocdepth}{1}
\tableofcontents
\vspace{6ex}

\section{Regression Analysis: Introduction}
\label{sec:orga7bb3be}
\begin{itemize}
\item How can we make predictions about real-world quantities, like sales or life
expectancy?
\item Most often in real world applications we need to understand how one variable
is determined by \textbf{a number of others}
\end{itemize}

For example:

\begin{itemize}
\item How does sales volume change with changes in price. How is this affected by
changes in the weather?
\item How is the interest rate charged on a loan affected by credit history and by
loan amount?
\end{itemize}

\begin{itemize}
\item We already used correlation coefficient to look at the relationship between 
\emph{two} variables, but \ldots{}
\item We cannot say that the correlation coefficient is a "pure" effect of
one variable's change on another variable

\begin{itemize}
\item e.g., What if \(x_{1}\) (e.g., price) and \(x_{2}\) (e.g., advertising) are also correlated?

\begin{center}
\begin{tabular}{lrrr}
\(\rho\) & Sales & Price & Advertising\\
Sales & 1 & -0.8 & 0.8\\
Price &  & 1 & -0.9\\
Advertising &  &  & 1\\
\end{tabular}
\end{center}
\end{itemize}
\end{itemize}
\subsection{Regression Analysis}
\label{sec:orge70bc4a}
\begin{itemize}
\item Let's you

\begin{itemize}
\item Discover relationship between a dependent variable (\(y\)) and 
multiple independent variables (\(x\)'s) jointly
\item Identify and measure each independent variable (\(x\))'s impact on 
\(y\) separately 
\begin{itemize}
\item While \emph{controlling for} (holding others constant) other variables
\end{itemize}
\end{itemize}
\end{itemize}

\subsection{Relationship between \(x\) and \(y\)}
\label{sec:orgebe5628}
\begin{itemize}
\item Essentially, we want to figure out the relationship between \(y\)
(dependent variable) and \(x\) (independent, explanatory) variables:

\[ y_i = f(x_{1i}, x_{2i}, \cdots) \]

\begin{itemize}
\item Where

\begin{itemize}
\item \(i\): \(i\)'th observation, \(n\): total number of observations
\item \(y_i\): dependent variable
\item \(x_{ki}\): \(i\)'th observation of \(k\)'th independent
(explanatory) variable
\item \(f(\cdot)\): the function specifying the relationship between \(y\)
and \(x\)
\end{itemize}
\end{itemize}
\end{itemize}

\begin{itemize}
\item e.g.,

\[ \underbrace{y_{i}}_{\text{Sales}_i} = f(\underbrace{x_{
  1i}}_{\text{Price}_i}, \underbrace{x_{2i}}_{\text{Promotion}_i}) \]

\item We basically want to know what \(f()\) is. For example,

\[ y_{i} = f(x_{1i}, x_{2i}) = 1 + 2 \times x_{1i}  + 3 \times x_{2i} \]
\end{itemize}
\subsection{Functional Form of \(f\): Linear Regression}
\label{sec:orgcacf982}
\begin{itemize}
\item In linear regression, we assume the dependent variable (\(y_{i}\)) to be a
linear function of independent (or explanatory) variables (\(x_{k}\)'s),
coefficients (\(\beta_{k}\)'s) and the error term (\(\varepsilon_{i}\)):

\[  y_{i} = \beta_0 + \beta_1 x_{1i} + \beta_2 x_{2i} + \varepsilon_{i} \]
\end{itemize}

\begin{itemize}
\item Where

\begin{itemize}
\item \(\beta_k\): coefficient for independent variable \(x_k\), which
represents the importance of \(x_{k}\) in \(y\)
\item \(\varepsilon_{i}\): the remaining part (error)

\begin{itemize}
\item Unpredictable with \(x\)'s
\begin{itemize}
\item e.g., random-walk of stock prices
\end{itemize}
\end{itemize}
\end{itemize}
\end{itemize}

\begin{mdframed}[frametitle={}]
Note that \(\beta_0\) is by itself since it corresponds to the constant term. That is, it 
represents the intercept, and you can think of it as \(x_{0i}\) being 1 everywhere 
(\(\beta_0 \times 1 = \beta_0\)).
\end{mdframed}
\section{Regression Analysis: Estimation}
\label{sec:orgd6312f0}
\subsection{Estimation: Ordinary Least Squares (OLS)}
\label{sec:org8b64bff}
\begin{itemize}
\item Again the regression model is:

\[  y_{i} = \beta_0 + \beta_1 x_{1i} + \varepsilon_{i} \]
\end{itemize}

\begin{itemize}
\item You can rearange terms
and characterize the error by:

\[  \varepsilon_{i} =  y_{i} -  \beta_0 - \beta_{1} x_{1i} \]

\item Since \(y_i\) and \(x_{1i}\) are data so they do not vary. Then, as you
change \(\beta_0\) and \(\beta_1\), \(\varepsilon_{i}\) will change.
\end{itemize}

Estimation Objective: \textbf{minimize} the sum of squared errors across all
observations:

\[ \sum_{i=1}^n\varepsilon^2_i = \varepsilon^2_1 + \varepsilon^2_2 + \cdots +
   \varepsilon^2_{n-1} + \varepsilon^2_{n} \]

\begin{itemize}
\item We want to find values of \(\beta_0\) and \(\beta_1\) that minimize the
sum of squared errors
\end{itemize}

Fortunately, we have analytical solutions for the \(\beta_0\) and \(\beta_1\):

\[ \widehat{\beta_1} = \frac{ \sum_{i=1}^{n}
   (x_{1i}-\bar{x}_1)(y_{i}-\bar{y}) }{ \sum_{i=1}^{n} (x_{1i}-\bar{x})^2 }
   \]



\[ \widehat{\beta_0}  = \bar{y} - \widehat\beta\,\bar{x}_{1} \]

\begin{itemize}
\item Where \(\widehat{\beta}_{k}\): estimate (actual number) of coeffcient \(\beta_k\)
\end{itemize}
\section{Interpretation of Regression Results: Fit (Model Level)}
\label{sec:org8330a9f}
\begin{itemize}
\item Remember how we estimate coefficients (\(\beta_k\)'s)?
\item \(\beta_k\) which minimize the sum of squared errors are the
estimates, \(\widehat\beta_k\)
\item How do we measure how well our model performs?
\end{itemize}

\subsection{Sum of Squares}
\label{sec:orgedb92b6}
\begin{description}
\item[{Total sum of squares (\(SS_{total}\))}] \quad

\[  SS_{tot} = \sum_{i=1}^n (y_i - \bar{y})^2 \]

\begin{itemize}
\item How much variation is in \(y\) (It's similar to variance)
\end{itemize}
\end{description}

\begin{description}
\item[{Sum of Squared Errors (\(SS_{error}\))}] \quad

\iffalse
\[ \begin{aligned}
      SS_{err} &= \varepsilon^2_1 + \varepsilon^2_2 + \cdots +
                   \varepsilon^2_{n-1} + \varepsilon^2_{n} = \sum_{i=1}^n \varepsilon^2_i 
   \end{aligned} \]

\[ = \sum_{i=1}^n \left\{ y_i - \underbrace{(\beta_0 + \beta_1 x_{i})}_{\text{predicted}} \right\}^2 \]
\fi

\[ \begin{aligned}
      SS_{err} &= \varepsilon^2_1 + \varepsilon^2_2 + \cdots +
                   \varepsilon^2_{n-1} + \varepsilon^2_{n} = \sum_{i=1}^n \varepsilon^2_i 
      = \sum_{i=1}^n \left\{ y_i - \underbrace{(\beta_0 + \beta_1 x_{i})}_{\text{predicted}} \right\}^2
   \end{aligned} \]
\end{description}
\subsection{Sum of Squared Errors (Residuals)}
\label{sec:orgaf2d2a6}
\begin{itemize}
\item \(SS_{error}\) is a measure of how wrong the regression estimates will be
overall
\item \(SS_{error}\) is a measure of variance
\item \(y_i\) is sometimes higher, sometimes lower than the regression line
\item Actual value of \(y_i\) varies because unobserved factors and randomness
\item The regression can never be a perfect predictor
\end{itemize}
\subsection{How well does regression fit?}
\label{sec:org0551e7c}
\begin{itemize}
\item We can use these to construct a value which represents:

\begin{itemize}
\item what \% of total variance do we explain with our model?

\[ \Rightarrow \dfrac{\text{explained variance}}
       {\text{total variance } (SS_{total})}
    \]

\item which can also be represented as

\[
       1 - \dfrac{\text{unexplained variance } (SS_{error})}
       {\text{total variance } (SS_{total})}
    \]
\end{itemize}
\end{itemize}

\subsubsection{\(R^2\)}
\label{sec:orgef26a00}

\begin{description}
\item[{\(R^2\)}] the percentage of variance in the dependent variable (\(y\))
explained by the independent variables (\(x\)'s):

\[ R^2 = 1 - \dfrac{SS_{error}}{SS_{total}} \]
\end{description}

\begin{itemize}
\item \(R^2\) is between 0 and 1 (0\% to 100\%)
\end{itemize}
\section{Interpretation of Regression Results: Coefficients}
\label{sec:org190c58d}

\begin{itemize}
\item \(\hat{\beta}_1\) (estimated coefficient for \(x_1\)): How much the
 {\bf dependent variable (\( y \))} is expected to change when the
 {\bf independent variable (\( x_{1} \))} increases by
 {\bf one} unit
\end{itemize}

\begin{itemize}
\item Suppose we have \(x_{1}\)'s value as 50, and \(\hat\beta_0 = 1\) and \(\hat\beta_1 = 3\). Then, the predicted \(y\) value is:

\[ \underbrace{\hat\beta_0}_{1} + \underbrace{\hat\beta_1}_{3} \times 50 = 151 \]

\item If we increase \(x_{1}\) by 1:

\[ \underbrace{\hat\beta_0}_{1} + \underbrace{\hat\beta_1}_{3} \times (50 + 1) = 154 \]

\item That is, \(y\) increases by \(\hat\beta_1\) when we increase
\end{itemize}
'eee' is not recognized as an internal or external command,
operable program or batch file.
\begin{itemize}
\item Mathematically,

\[  \dfrac{\partial y}{\partial x} = \dfrac{\partial ( \beta_0 + \beta_1 x)}{\partial x} = \beta_1 \]
\end{itemize}
\section{Multiple Regression}
\label{sec:orge809cd4}
\subsection{Multiple Regression}
\label{sec:org2bbb0e9}
\begin{itemize}
\item Sales vs. Promotion Discount is an example of simple linear regression
\item But sales of a brand depend upon many things
\begin{itemize}
\item TV Ads, In-store promotions, Coupons etc \ldots{}
\end{itemize}

\item When many things vary at the same time, it is hard to visually see the
impact of each factor
\item Multiple regression lets you look at an isolated effect of one variable
\end{itemize}

\[ y_{i} = \beta_0 + \beta_1 x_{i, 1} + \cdots + \beta_k x_{i, k} + \cdots +
\beta_K x_{i, K} + \varepsilon_{i} \]

\begin{itemize}
\item Interpretation of \(\hat{\beta}_k\): \uline{holding other variables constant}, the
change in \(y\) if you increase \(x_k\) by 1 unit

\item Just like the simple regression, mathematically,

\[ \dfrac{\partial y}{\partial x_k} = \dfrac{\partial ( \beta_0 + \beta_1 x_1 + \cdots + \beta_k
  x_k + \cdots + \beta_K x_K) }{\partial x_k } = \beta_k. \]
\end{itemize}

\subsection{\(R^2\) and Adjusted \(R^2\)}
\label{sec:org0d168a0}
\begin{itemize}
\item Recall

\[
     R^2 = 1 - \dfrac{\text{unexplained variance } (SS_{error})}
           {\text{total variance } (SS_{total})} = 1 - \dfrac{SS_{error}}{SS_{total}}
  \]

\item \(R^2\) is between 0 and 1 (0\% to 100\%)
\end{itemize}
\subsubsection{\(R^2\) in multiple regression}
\label{sec:org4cff366}
\begin{itemize}
\item \(R^2\) \textbf{always} becomes larger when we add more
independent variables
\item So we CANNOT use \(R^2\) to compare the fit of two different regressions
with different numbers of independent variables
\end{itemize}
\subsubsection{Adjusted \(R^{2}\)}
\label{sec:org444853b}
\begin{itemize}
\item We use \textbf{adjusted} \(R^2\) to compare regressions with
different numbers of independent variables

\[  R^2_{adj} = 1 - \left\{ \dfrac{SS_{error}}{SS_{total}} \times
      \dfrac{n-1}{n-K-1} \right\} \]

\begin{itemize}
\item \(n\): number of observations
\item \(K\): number of independent (\(x\)) variables included in the model
\end{itemize}
\end{itemize}

\iffalse
\[  R^2_{adj} = 1 - \left\{ \dfrac{SS_{error}}{SS_{total}} \times
    \dfrac{n-1}{n-K-1} \right\} \]
\fi

\begin{itemize}
\item Basically, you give a little bit of penalty for higher \(K\)
\item A variable needs to reduce \(SS_{error}\) significantly to overcome the
penalty
\end{itemize}

\begin{itemize}
\item Occam's razor: 

"Among competing hypotheses, the one with the fewest assumptions should be selected"
\end{itemize}

\begin{itemize}
\item Albert Einstein:

"Everything should be made as simple as possible, but no simpler"
\end{itemize}
\section{Running Regression Analysis in Excel}
\label{sec:org5fa5e51}
\iffalse
\begin{itemize}
\item Mac Users: please install \textbf{Excel 2016 for Mac}
\end{itemize}
\fi

\begin{mdframed}
\textbf{Note for Mac users}

\begin{itemize}
\item It is required that you have \textbf{Excel 2016 for Mac} installed in your device
for this course, as \texttt{Data Analysis ToolPak} and \texttt{Solver} add-ins are not
available in previous versions of Microsoft Excel for Mac. It still has some
limitations comparing to the Windows version, but you will be able to
accomplish most of tasks.
\end{itemize}
\end{mdframed}

\subsection{Activating \texttt{Analysis ToolPak} and \texttt{Solver} in Excel}
\label{sec:orge7b758a}
\begin{itemize}
\item You have to activate the \texttt{Analysis ToolPak} and \texttt{Solver}. (You only need the former
for running Regression, but we may need the latter as well)
\end{itemize}
\subsubsection{On Windows}
\label{sec:orgb8d21f4}
\begin{enumerate}
\item Click \texttt{File} tab -> Click \texttt{Options} -> \texttt{Add-Ins} category.
\item Near the bottom of the \texttt{Excel Options} dialog box, make sure that \texttt{Excel
   Add-ins} is selected in the \texttt{Manage} box, and then click \texttt{Go...}.
\item In the \texttt{Add-Ins} dialog box, select the check boxes for \texttt{Analysis ToolPak}
and \texttt{Solver Add-in}, and then click \texttt{OK}.
\item If Excel displays a message that states it can't run this add-in and
prompts you to install it, click \texttt{Yes} to install the add-ins.
\item On the \texttt{Data} tab, note that an \texttt{Analyze} group has been added. This group
contains command buttons for \texttt{Data Analysis} and for \texttt{Solver}.
\end{enumerate}
\subsubsection{On Mac}
\label{sec:orgdfaee7f}
\begin{enumerate}
\item Go to the \texttt{Tools} menu, select \texttt{Add-ins}
\item Check \texttt{Analysis ToolPak} and \texttt{Solver Add-in} and then click \texttt{OK}
\end{enumerate}
\subsection{Running Regression Analysis in Excel}
\label{sec:org261235a}
\begin{itemize}
\item Click \texttt{Data} -> \texttt{Data Analysis} -> Select \texttt{Regression} -> \texttt{OK}

\item \texttt{Input Y Range}: Specify the range of cells that contain values for the
dependent variable (\(y\)), \uline{including the variable label} (the first row)

\item \texttt{Input X Range}: Specify the range of cells that contain values for the
independent (predictor) variable (\(x\)), \uline{including the variable label}
(the first row)

\begin{itemize}
\item For multiple regression, all \(x\) variables must be adjacent to each other. (no gaps)

\item Tip: You can use \texttt{Shift + Down Arrow} and \texttt{Ctrl + Shift + Down Arrow} to
select a range of data conveniently
\end{itemize}

\item Make sure \texttt{Labels} checkbox is checked
\item Click \texttt{OK} (it will put the results on a new sheet)
\end{itemize}
\end{document}