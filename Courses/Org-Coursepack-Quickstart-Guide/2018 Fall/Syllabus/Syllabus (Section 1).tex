% Intended LaTeX compiler: pdflatex
\documentclass[article,letterpaper,times,10pt,listings-bw,microtype]{scrartcl}
\usepackage[utf8]{inputenc}
\usepackage[T1]{fontenc}
\usepackage{graphicx}
\usepackage{grffile}
\usepackage{longtable}
\usepackage{wrapfig}
\usepackage{rotating}
\usepackage[normalem]{ulem}
\usepackage{amsmath}
\usepackage{textcomp}
\usepackage{amssymb}
\usepackage{capt-of}
\usepackage{hyperref}
\usepackage[onehalfspacing]{setspace} \usepackage[T1]{fontenc} \usepackage{mathpazo} \usepackage[scaled=0.85]{helvet} \usepackage{courier} \usepackage{geometry} \geometry{left=1in,right=1in,top=1in,bottom=1in} \usepackage[framemethod=default]{mdframed}
\author{Your Name Your Name}
\date{Fall 2018}
\title{COURSE 0000-01 Syllabus}
\hypersetup{
 pdfauthor={Your Name Your Name},
 pdftitle={COURSE 0000-01 Syllabus},
 pdfkeywords={},
 pdfsubject={Org-Coursepack quickstart guide School specific teaching materials},
 pdfcreator={Emacs 26.1 (Org mode 9.1.14)}, 
 pdflang={English}}
\begin{document}

\definecolor{SchoolColor}{RGB}{70,130,180}

\begin{center}
{\color{SchoolColor}{\Large
\textbf{School Name}
}}
\end{center}

\begin{center}
{\color{SchoolColor}{
\textbf{COURSE 0000-01}

\textbf{Org-Coursepack Quickstart guide}

\textbf{Fall 2018}
}}
\end{center}

\vspace{5 mm}

\begin{center}
\begin{tabular}{llll}
\textbf{Instructor:} & Your Name & \textbf{Office Phone:} & (000) 000-0000\\
\textbf{Office:} & BLDG 100 & \textbf{E-mail:} & YourEmail\\
\textbf{Office Hours:} & Tue 3:30-4:30pm & \textbf{Course Site:} & \textbf{\href{https://github.com}{github.com}}\\
\textbf{Class Meeting Day \& Time:} & Tue/Thurs, 9:30a-10:45 & \textbf{Class Location:} & BLDG 100\\
\end{tabular}
\end{center}
\section*{Course Description}
\label{sec:orgba742b8}
This is a course intended as the quickstarter guide for \texttt{Org-Coursepack}. We
aim to equip students with the basic usage of \texttt{Org-Coursepack}.
\section*{Course Prerequisites}
\label{sec:org4c7a1f4}
\begin{itemize}
\item Course Prerequisites here.
\end{itemize}
\section*{Student Learning Objectives}
\label{sec:org19b521a}
As the result of this course, students should be able to:

\begin{itemize}
\item Student Learning Objectives here.
\end{itemize}
\section*{Course Material}
\label{sec:org1858e2f}
\begin{itemize}
\item Course Materials here.
\end{itemize}
\section*{Grading}
\label{sec:org2195b5d}
Grading information here.
\subsection*{Grade Cutoff Percentage}
\label{sec:org9788cad}
The following table provides the preliminary guidelines for assigning your
letter grade:

\begin{center}
\begin{tabular}{llllllllllll}
\hline
Grade & A & A- & B+ & B & B- & C+ & C & C- & D+ & D & D-\\
\hline
Cutoff Pct & 93\% & 90\% & 87\% & 83\% & 80\% & 77\% & 73\% & 70\% & 67\% & 63\% & 60\%\\
\hline
\end{tabular}
\end{center}
\section*{Specific Course Policies}
\label{sec:orga1a71ad}
Specific Course Policies here.
\section*{School-Specific Policies}
\label{sec:orge295f9e}
School-specific policies here.
\clearpage
\section*{Class Schedule}
\label{sec:org1e71c80}
\begin{center}
\begin{tabular}{lrl}
Date & Class & Topic\\
\hline
Tue, Aug 28, 2018 & 1 & Introduction\\
Thu, Aug 30, 2018 & 2 & Second Lecture\\
Tue, Sep 04, 2018 & 3 & Third Lecture\\
 &  & \uline{Assignment 1 Due}\\
Thu, Sep 06, 2018 & 4 & \uline{Exam 1}\\
Wed, Nov 21, 2018--Sun, Nov 25, 2018 &  & \textbf{Thanksgiving Holiday}\\
Sun, Dec 16, 2018  1:00PM - 4:00PM &  & \uline{Final Exam}\\
\end{tabular}
\end{center}

\begin{mdframed}[style=exampledefault, frametitle={Disclaimer}]
\begin{itemize}
\item The class schedule is subject to change (except for the exam dates)
\end{itemize}
\end{mdframed}
\end{document}