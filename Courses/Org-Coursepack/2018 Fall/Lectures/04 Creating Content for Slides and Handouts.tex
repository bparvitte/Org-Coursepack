% Intended LaTeX compiler: pdflatex
\documentclass[10pt,article]{article}
\usepackage[utf8]{inputenc}
\usepackage[T1]{fontenc}
\usepackage{graphicx}
\usepackage{grffile}
\usepackage{longtable}
\usepackage{wrapfig}
\usepackage{rotating}
\usepackage[normalem]{ulem}
\usepackage{amsmath}
\usepackage{textcomp}
\usepackage{amssymb}
\usepackage{capt-of}
\usepackage{hyperref}
\usepackage{titling} \posttitle{\par\end{center}} \setlength{\droptitle}{-30pt} \usepackage{multicol} \setlength{\columnsep}{1cm} \usepackage[T1]{fontenc} \usepackage[utf8]{inputenc} \renewcommand{\contentsname}{Table of Contents / Agenda} \usepackage[letterpaper,left=1in,right=1in,top=0.7in,bottom=1in,headheight=23pt,includehead,includefoot,heightrounded]{geometry} \usepackage{fancyhdr} \pagestyle{fancy} \fancyhf{} \cfoot{\thepage} \usepackage{mathpazo} \usepackage[scaled=0.85]{helvet} \usepackage{courier} \usepackage[onehalfspacing]{setspace} \usepackage[framemethod=default]{mdframed} \usepackage{wrapfig} \usepackage{booktabs} \usepackage[outputdir=Lectures]{minted}
\setcounter{secnumdepth}{3}
\date{\vspace{-6ex}}
\title{Class 4: Creating Content for Slides and Handouts}
\hypersetup{
 pdfauthor={},
 pdftitle={Class 4: Creating Content for Slides and Handouts},
 pdfkeywords={},
 pdfsubject={Org-Coursepack-Documentation: A Comprehensive Course Template for Teaching with Org Mode School specific teaching materials},
 pdfcreator={Emacs 25.3.1 (Org mode 9.1.13)}, 
 pdflang={English}}
\begin{document}

\maketitle
\lhead{ ORG 0000 \\ Joon H. Ro \& Jae-Eun Namkoong } 
\rhead{ Class 4 \\ 2018-09-06 Thu} 
\thispagestyle{fancy}

\setcounter{tocdepth}{1}
\tableofcontents
\vspace{6ex}

\section{Creating Slides and Handouts: Introduction}
\label{sec:org611f9f9}
Content is created using the Org markup language. General information on how
to use the markeup language can be found in the \href{https://orgmode.org/manual/Markup.html}{Org manual}. Hence,
we focus on common slide- and handout-related tasks that cannot be achieved
with common Org markup language. These tasks require either raw HTML tags or
\LaTeX{} code, or custom Emacs lisp code. For tasks that require custom Emacs
Lisp code, users can put code snippets we present here into their Emacs init
file.

\begin{mdframed}
We use reveal.js as the default slide format, and while Beamer slides are not
completely supported by Org-Coursepack yet, in some cases we do provide the same
functionality for Beamer as well as reveal.js. In those cases we have a
note describing how to achieve the functionality in Beamer.
\end{mdframed}
\section{Using Raw HTML Tags and \LaTeX{} Code}
\label{sec:orgb0bc942}
Directly quoting raw HTML tags and \LaTeX{} code allows users to have granular
control over how contents are presented. Such quotes will be only included in
their corresponding outputs that are HTML-based (e.g., reveal.js) and
\LaTeX{}-based, respectively. Hence, to understand the information that follow,
the readers should be familiar with the information in the Org manual on
\href{https://orgmode.org/manual/Quoting-HTML-tags.html}{Quoting HTML tags} and \href{https://orgmode.org/manual/Quoting-LaTeX-code.html}{Quoting \LaTeX{} code}.

\begin{mdframed}
\textbf{Inline Raw Code}

To use Org macros with raw \LaTeX{} code (e.g., surround a macro with
\LaTeX{} code), use \texttt{@@latex:your code here@@} (same grammar
applies to HTML as well) like the following:

\begin{verbatim}
  @@latex:{\small@@ {{{COURSE}}} @@latex:}@@
\end{verbatim}
\end{mdframed}
\section{Features for Both Slides and Handouts}
\label{sec:org3b490c1}
\subsection{Specifying Attributes}
\label{sec:org0af6cac}
Org mode allows users to specify attributes to raw HTML tags or \LaTeX{} code
using \texttt{\#+ATTR\_FORMAT:} grammar. For example, the following shows how to 
specify the width of an image.

\begin{verbatim}
#+ATTR_HTML: :width 80%
[[/img/image.png]]
\end{verbatim}

For more information, see the \href{https://orgmode.org/worg/org-tutorials/images-and-xhtml-export.html}{tutorial on Images and XHTML export}.

\subsection{Changing Font Sizes}
\label{sec:org72b2ec1}
One of the frequently used use cases of raw HTML or \LaTeX{} code in Org markup
is changing the font size of a specific text. 

For example, to apply a smaller
font size in HTML outputs, the user can use the following code.

\begin{minted}[]{text}
  #+HTML: <span style=font-size:20pt>
  Content with smaller font
  #+HTML: </span>
\end{minted}

In \LaTeX{} handouts, the user can use the code below.

\begin{minted}[]{text}
  #+LATEX: {\small
  Content with smaller font
  #+LATEX: }
\end{minted}

Since raw code that is irrelevant to the specific output format (e.g.,
HTML codes in a \LaTeX{} output) will be ignored, users can safely combine 
HTML and LateX codes and use them together like so:

\begin{minted}[]{text}
    #+LATEX: {\small
    #+HTML: <span style=font-size:20pt>
    Content with smaller font
    #+HTML: </span>
    #+LATEX: }
\end{minted}
\subsection{Using a Dummy Heading}
\label{sec:org4f171d5}
Instructors may want the option to present content of a tree without its heading. To do so, follow the instructions at
\url{https://orgmode.org/worg/org-hacks.html\#ignoreheadline}. Specifically, include the
following in your init file, and any header with the \texttt{:ignore:} tag will not be printed in exported outputs.

\begin{minted}[]{common-lisp}
(require 'ox-extra)
(ox-extras-activate '(ignore-headlines))
\end{minted}
\section{Slides: Features for reveal.js}
\label{sec:org88aafb2}
\begin{mdframed}
Note that reveal.js is HTML-based, so any raw HTML tags (e.g., via \texttt{\#+HTML:})
or attributes (e.g., via \texttt{\#+ATTR\_HTML:}) will be applied to reveal.js as well
as all HTML-based output formats.  For codes that are only for reveal.js, one
should use \texttt{\#+REVEAL:} (\texttt{\#+ATTR\_REVEAL:}) instead of \texttt{\#+HTML:}
(\texttt{\#+ATTR\_HTML:}) to avoid unnecessary tags being exported.
\end{mdframed}

\subsection{List Fragments}
\label{sec:org3b19df4}
One can easily obtain list fragments (make items in the list appear
sequentially) using reveal.js. Simply add \texttt{\#+ATTR\_REVEAL: :frag
(appear)} before the list. See the example below.

\begin{minted}[]{text}
  #+ATTR_REVEAL: :frag (appear)
  - I appear first.
  - I appear second.
  - I appear third.
\end{minted}

\begin{mdframed}
Similarly, a list fragment can be obtained on the Beamer output by including
\texttt{\#+ATTR\_BEAMER: :overlay <+->} before the list.
\end{mdframed}

\subsection{Splitting slides}
\label{sec:org157bd67}
To split content into multiple slides, insert the following code between the areas where you want the split to happen.

\begin{minted}[]{text}
  #+REVEAL: split
\end{minted}

\begin{mdframed}
Similarly, a frame break can be inserted in Beamer by using \texttt{\#+BEAMER:
\textbackslash{}framebreak}.
\end{mdframed}

\subsection{Embedding Youtube videos}
\label{sec:orge84475f}
One can use the following example to embed a YouTube video in a slide. The
example specifies at which points of the video the viewing will start
(1 second in) and end (60 seconds in).
{\small
\begin{minted}[]{text}
  #+BEGIN_EXPORT HTML
  <iframe width="1066" height="570"
  src="https://www.youtube.com/embed/SzA2YODtgK4?start=01&end=60" allowfullscreen>
  </iframe>
  #+END_EXPORT
\end{minted}
}
\subsection{Speaker Notes}
\label{sec:org7ffce18}
An instructor may create a speaker note that accompanies a lecture
slide. reveal.js will display the speaker note in a separate browser
window. To create a speaker note, use a \texttt{NOTES} block as shown in the example
below.

\begin{minted}[]{text}
#+BEGIN_NOTES
- This is a speaker note.
#+END_NOTES
\end{minted}

The following code needs to be inserted in the init file to hide speaker
notes in \LaTeX{} and HTML output formats.

\begin{mdframed}
Note that using the example code below will also make speaker notes appear
properly on Beamer.
\end{mdframed}

{\small
\begin{minted}[]{common-lisp}
(defun my/remove-NOTES-blocks (text backend info)
  "Filter special blocks from latex export."
  (cond
   ((eq backend 'latex)
    (if (string/starts-with text "\\begin{NOTES}") ""))
   ((eq backend 'html)
    (if (string/starts-with text "<div class=\"NOTES\">") ""))
   ((eq backend 'beamer)
    (let ((text (replace-regexp-in-string "\\\\begin{NOTES}" "\\\\note{" text)))
      (replace-regexp-in-string "\\\\end{NOTES}" "}" text)))
))

(eval-after-load 'ox '(add-to-list
                       'org-export-filter-special-block-functions
                       'my/remove-NOTES-blocks))

\end{minted}
}
\section{Handouts: Features for \LaTeX{}}
\label{sec:orgfc76de7}
The features introduced in this section are readily available, as the
necessary items in the \LaTeX{} preamble enabling the features are already
specified in the properties of the \texttt{Lectures} subtree in the semester Org
files of the Org-Coursepack.
\subsection{Inserting Boxed Paragraphs}
\label{sec:orge9be1e0}
With the \texttt{mdframed} block, users can easily create boxed paragraphs in
handouts. The example below shows the code for the box and what the box will
look like in the handout. Note that the title of the box is written in bold
instead of using \texttt{\#+ATTR\_LATEX: :options [frametitle=\{Title of the box\}]}
option, so the title gets printed in both reveal.js and \LaTeX{} outputs.

\begin{minted}[]{text}
#+BEGIN_mdframed
*Title of the box*

Content of the box
#+END_mdframed
\end{minted}

\begin{mdframed}
\textbf{Title of the box}

Content of the box
\end{mdframed}
\subsection{Organizing Content in Multiple Columns}
\label{sec:org74a0848}
One can easily make parts of the handout multi-column. The example
below shows the code for creating two columns and what that will look like in
the handout.

\begin{minted}[]{text}
  #+LATEX: \begin{multicols}{2}
  This is content in the first column.
  This is content in the first column.
  This is content in the first column.

  This is content in the second column.
  This is content in the second column.
  This is content in the second column.
  #+LATEX: \end{multicols}
\end{minted}

\begin{multicols}{2}
This is content in the first column.
This is content in the first column.
This is content in the first column.

This is content in the second column.
This is content in the second column.
This is content in the second column.
\end{multicols}
\end{document}