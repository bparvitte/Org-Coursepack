% Intended LaTeX compiler: pdflatex
\documentclass[10pt,article]{article}
\usepackage[utf8]{inputenc}
\usepackage[T1]{fontenc}
\usepackage{graphicx}
\usepackage{grffile}
\usepackage{longtable}
\usepackage{wrapfig}
\usepackage{rotating}
\usepackage[normalem]{ulem}
\usepackage{amsmath}
\usepackage{textcomp}
\usepackage{amssymb}
\usepackage{capt-of}
\usepackage{hyperref}
\usepackage{titling} \posttitle{\par\end{center}} \setlength{\droptitle}{-30pt} \usepackage{multicol} \setlength{\columnsep}{1cm} \usepackage[T1]{fontenc} \usepackage[utf8]{inputenc} \renewcommand{\contentsname}{Table of Contents / Agenda} \usepackage[letterpaper,left=1in,right=1in,top=0.7in,bottom=1in,headheight=23pt,includehead,includefoot,heightrounded]{geometry} \usepackage{fancyhdr} \pagestyle{fancy} \fancyhf{} \cfoot{\thepage} \usepackage{mathpazo} \usepackage[scaled=0.85]{helvet} \usepackage{courier} \usepackage[onehalfspacing]{setspace} \usepackage[framemethod=default]{mdframed} \usepackage{wrapfig} \usepackage{booktabs} \usepackage[outputdir=Lectures]{minted}
\setcounter{secnumdepth}{3}
\date{\vspace{-6ex}}
\title{Class 5: Exporting Slides and Handouts}
\hypersetup{
 pdfauthor={},
 pdftitle={Class 5: Exporting Slides and Handouts},
 pdfkeywords={},
 pdfsubject={Org-Coursepack-Documentation: A Comprehensive Course Template for Teaching with Org Mode School specific teaching materials},
 pdfcreator={Emacs 25.3.1 (Org mode 9.1.13)}, 
 pdflang={English}}
\begin{document}

\maketitle
\lhead{ ORG 0000 \\ Joon H. Ro \& Jae-Eun Namkoong } 
\rhead{ Class 5 \\ 2018-09-11 Tue} 
\thispagestyle{fancy}

\setcounter{tocdepth}{1}
\tableofcontents
\vspace{6ex}

\section{Exporting Content in Org Mode}
\label{sec:org8ce5740}
Org mode provides powerful export functionalities, which enable users to
convert Org markup content to a variety of other formats. The outputs have proper
formatting while maintaining the original structure and markup. 
The Org-Coursepack is set up to use reveal.js for slides and \LaTeX{} for
handouts. Users can adapt the current setting to accomodate other output
formats.

General information about exporting can be found in the \href{https://orgmode.org/manual/Exporting.html}{Org manual}. Hence,
this lecture focuses on describing how exporting is set up in the
Org-Coursepack, introducing pre-built export functionalities of the
Org-Coursepack, and offering useful exporting tips for instructors.
\subsection{Setting Export Scope to \texttt{Subtree}}
\label{sec:org556af86}
The export command, \texttt{org-export-dispatch} (\texttt{C-c C-e}), takes the user
to the \texttt{Org Export Dispatcher} interface. Here, the user can select
whether to export the buffer (i.e., the whole file) or only a
subtree. An instructor may use the former to create a course
booklet and the latter to export slides for a full or part of a
lecture. The default scope of Org mode is \texttt{Buffer}, but a user can put the following snippet in
the Emacs init file to set the default scope to \texttt{Subtree}.

\begin{minted}[]{common-lisp}
  ;; set the default export scope to subtree
  (setf org-export-initial-scope 'subtree)
\end{minted}
\subsection{Export Settings}
\label{sec:orgd059bab}
Org mode allows users to specify \href{https://orgmode.org/manual/Export-settings.html}{export settings} at both buffer and subtree levels.
\subsubsection{Buffer-Level Settings}
\label{sec:org4ec5e41}
For buffer-level settings, one can use the \texttt{\#+OPTIONS:} statement. For
example, including the following line in an Org file will include a table
of contents for that file in the exported document:

\begin{verbatim}
#+OPTIONS: toc:t
\end{verbatim}

Similarly including the following line will add numbers in front of the headings.

\begin{verbatim}
#+OPTIONS: num:t
\end{verbatim}

\subsubsection{Subtree-Level Settings}
\label{sec:orgafd1342}
Export settings can be specified at the subtree level with
\texttt{:PROPERTIES:}. Simply add \texttt{:EXPORT\_:} as a prefix to each option. For
example, the title of the document can be set with
\texttt{:EXPORT\_TITLE:}. To specify multiple settings (e.g., items for \LaTeX{} preamble),
one can add \texttt{+} to the property name to append an additional value.  For example,

\begin{verbatim}
:PROPERTIES:
:EXPORT_LATEX_HEADER+: \usepackage{titling}
:EXPORT_LATEX_HEADER+: \usepackage{multicol}
:END:
\end{verbatim}
\section{Clickable Links and Keybindings for Exporting}
\label{sec:org00bbfc3}
\subsection{Buttons}
\label{sec:orga4b9d8d}
We include export buttons (in the form of links written in Emacs-lisp) in the
properties of any exportable subtree in the Org-Coursepack (e.g.,
lectures, syllabi, and exams).

For example, each lecture headline comes with the \texttt{:EXPORT\_TO:}
property, which includes buttons such as \texttt{reveal.js} and
\texttt{LaTeX}. These buttons will export files to their respective format,
using \texttt{Subtree} as the export scope.

After exporting, users can click on buttons in the \texttt{:VIEW\_OUTPUT:}
property (e.g., \texttt{HTML} or \texttt{PDF}) to open the exported files.

\subsection{Key Bindings}
\label{sec:org15d9e48}
Using the command=org-export-dispatch= (\texttt{C-c C-e}) when exporting a
content allows users to later repeat the last export action for that
same content using the prefix argument (\texttt{C-u}). This is a convenient
feature when exporting the same content multiple times.

\vspace{5 mm}

\begin{mdframed}
Note that the cursor shoud be in the correct position (heading,
properties, or spaces between the heading and its first subheading) of
the subtree being exported. When using \texttt{C-u C-c C-e} for repeated
export, however, the cursor position does \emph{not} matter as long as the
same buffer which contains the last exported subtree is open.
\end{mdframed}

\vspace{5 mm}

To bind \texttt{C-u C-c C-e} to a key (\texttt{F5} in this example), include
the following Emacs-lisp code in into the init file:

\begin{minted}[]{common-lisp}
;; bind f5 to keyboard macro of export-last-subtree
(fset 'export-last-subtree
      "\C-u\C-c\C-e")

(eval-after-load "org"
  '(progn
     (define-key org-mode-map (kbd "<f5>") 'export-last-subtree)))
\end{minted}
\section{Exporting Slides: with reveal.js}
\label{sec:orgec5bdcb}
\begin{itemize}
\item See the \href{https://github.com/yjwen/org-reveal/}{org-reveal documentation} for instructions on installation and usage.
\end{itemize}
\section{Exporting Slides: with \LaTeX{}}
\label{sec:org59e7735}
\LaTeX{} export is extensively supported by Org mode. We refer users to the \href{https://orgmode.org/manual/LaTeX-export.html\#LaTeX-export}{Org
manual} for the in-depth instructions.

The following snippet shows the basic setup for our \LaTeX{} output,
where the \texttt{koma-article} class is added to \texttt{org-latex-classes} and the
\texttt{minted} package is used for syntax highlighting. Currently, Python is
the only language added to \texttt{org-latex-minted-langs}. Users can add to
\texttt{org-latex-minted-langs} any other languages they want processed with
the \texttt{minted} package.

\begin{mdframed}
Note that we manually added the \texttt{minted} package to \LaTeX{} preambles as opposed to adding
it to \texttt{org-latex-packages-alist}. This was to allow for flexible specifications of the \texttt{outputdir} option.
\end{mdframed}

{\small
\begin{minted}[]{common-lisp}
(eval-after-load 'ox '(require 'ox-koma-letter))

(eval-after-load 'ox '(add-to-list 'org-latex-classes
                                   '("koma-article"
                                     "\\documentclass{scrartcl}"
                                     ("\\section{%s}" . "\\section*{%s}")
                                     ("\\subsection{%s}" . "\\subsection*{%s}")
                                     ("\\subsubsection{%s}"
                                      . "\\subsubsection*{%s}")
                                     ("\\paragraph{%s}" . "\\paragraph*{%s}")
                                     ("\\subparagraph{%s}"
                                      . "\\subparagraph*{%s}"))) )

(require 'ox-latex)
(setq org-latex-listings 'minted)

(setq org-latex-pdf-process
      '("pdflatex -shell-escape -interaction nonstopmode -output-directory %o %f"
        "pdflatex -shell-escape -interaction nonstopmode -output-directory %o %f"))

(add-to-list 'org-latex-minted-langs '(python "python"))
\end{minted}
}

\subsection{Inserting a Page Break Before a Heading in \LaTeX{} Export}
\label{sec:org48c0563}
Users can add a page break in the \LaTeX{} export by inserting
\texttt{\#+LATEX: \textbackslash{}clearpage}. Importantly, adding the following code into the init file
automatically inserts a page break before any subtree that has a \texttt{:newpage:}
tag.

{\small
\begin{minted}[]{common-lisp}
(defun org/get-headline-string-element  (headline backend info)
  "Return the org element representation of an element.

  Won't work on ~verb~/=code=-only headers"
  (let ((prop-point (next-property-change 0 headline)))
    (if prop-point (plist-get (text-properties-at prop-point headline) :parent))))

(defun org/ensure-latex-clearpage (headline backend info)
  (when (org-export-derived-backend-p backend 'latex)
    (let ((elmnt (org/get-headline-string-element headline backend info)))
      (when (member "newpage" (org-element-property :tags elmnt))
        (concat "\\clearpage\n" headline)))))

(eval-after-load 'ox '(add-to-list
                       'org-export-filter-headline-functions
                       'org/ensure-latex-clearpage))
\end{minted}
}
\section{Selective Export}
\label{sec:org90a1fb8}
By using raw code and custom Emacs-lisp scripts, users can flexibly
choose which content to show/hide, depending on output format. For
example, instructors may want to show images in slides but not in
handouts, or they may want to include supplementary notes in handouts but not in slides.
\subsection{Tagging a Subtree as Slide or Handout Only}
\label{sec:org55e4ab9}
With the code below in your init file, you can use the \texttt{:slideonly:} or
\texttt{:handoutonly:} tags to selectively include a subtree in either a slide output
or handout output, respectively. Currently \LaTeX{} export is set as a handout output, and
reveal.js and beamer are set as slide outputs.

For example,

\begin{verbatim}
  * This subtree will only be exported in slide output      :slideonly:
  - Content
  * This subtree will only be exported in handout output    :handoutonly:
  - Content
\end{verbatim}

{\small
\begin{minted}[]{common-lisp}
  (defun org/parse-headings (backend)
    "Remove every headline with certain tags in the
    current buffer. BACKEND is the export back-end being used, as
    a symbol.
  
    "
  
    (if (member backend '(latex))
        (org-map-entries
         (lambda ()
           (progn
             (org-narrow-to-subtree)
             (org-cut-subtree)
             (widen)
             ))
         "+slideonly"))
  
    (if (member backend '(beamer reveal))
        (org-map-entries
         (lambda ()
           (progn
             (org-narrow-to-subtree)
             (org-cut-subtree)
             (widen)
             ))
         "+handoutonly"))
  
  )
  
  (add-hook 'org-export-before-parsing-hook 'org/parse-headings)
\end{minted}
}
\subsection{Hiding Specific Content}
\label{sec:orgd3b35ef}
To hide content when exporting to HTML-based format outputs (slides), use raw
HTML tags \texttt{<span hidden>} and \texttt{</span>}. See the example below.

{\small
\begin{minted}[]{html}
  #+REVEAL_HTML: <span hidden>
  This will not be shown in reveal.js output
  #+REVEAL_HTML: </span>
\end{minted}
}

Similarly, any content placed between \texttt{\textbackslash{}iffalse} and \texttt{\textbackslash{}fi} will not be
rendered in \LaTeX{} outputs (handouts). See the example below.

{\small
\begin{minted}[]{text}
  #+LATEX: \iffalse
  This will not be shown in LaTeX output
  #+LATEX: \fi
\end{minted}
}
\end{document}