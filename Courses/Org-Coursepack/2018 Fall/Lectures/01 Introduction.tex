% Intended LaTeX compiler: pdflatex
\documentclass[10pt,article]{article}
\usepackage[utf8]{inputenc}
\usepackage[T1]{fontenc}
\usepackage{graphicx}
\usepackage{grffile}
\usepackage{longtable}
\usepackage{wrapfig}
\usepackage{rotating}
\usepackage[normalem]{ulem}
\usepackage{amsmath}
\usepackage{textcomp}
\usepackage{amssymb}
\usepackage{capt-of}
\usepackage{hyperref}
\usepackage{titling} \posttitle{\par\end{center}} \setlength{\droptitle}{-30pt} \usepackage{multicol} \setlength{\columnsep}{1cm} \usepackage[T1]{fontenc} \usepackage[utf8]{inputenc} \renewcommand{\contentsname}{Table of Contents / Agenda} \usepackage[letterpaper,left=1in,right=1in,top=0.7in,bottom=1in,headheight=23pt,includehead,includefoot,heightrounded]{geometry} \usepackage{fancyhdr} \pagestyle{fancy} \fancyhf{} \cfoot{\thepage} \usepackage{mathpazo} \usepackage[scaled=0.85]{helvet} \usepackage{courier} \usepackage[onehalfspacing]{setspace} \usepackage[framemethod=default]{mdframed} \usepackage{wrapfig} \usepackage{booktabs} \usepackage[outputdir=Lectures]{minted}
\setcounter{secnumdepth}{3}
\date{\vspace{-6ex}}
\title{Class 1: Introduction}
\hypersetup{
 pdfauthor={},
 pdftitle={Class 1: Introduction},
 pdfkeywords={},
 pdfsubject={Fall 2018 Org-Coursepack School specific teaching materials},
 pdfcreator={Emacs 25.3.1 (Org mode 9.1.13)}, 
 pdflang={English}}
\begin{document}

\maketitle
\lhead{ ORG 0000 \\ Joon H. Ro \& Jae-Eun Namkoong } 
\rhead{ Class 1 \\ 2018-08-28 Tue} 
\thispagestyle{fancy}

\setcounter{tocdepth}{1}
\tableofcontents
\vspace{6ex}

\section{Introduction to Org-Coursepack}
\label{sec:orge309cc4}
The Org-Coursepack provides a template for developing and managing teaching
materials using \href{https://orgmode.org/manual/Export-settings.html}{Org mode}, a major mode in \href{https://www.gnu.org/software/emacs/manual/html\_node/emacs/Specifying-File-Variables.html\#Specifying-File-Variables}{GNU Emacs}.
\subsection{Advantages for Instructors}
\label{sec:orge28fa5c}
\begin{itemize}
\item First, Org mode and modular design allow for more effective and efficient
content creation.
\begin{itemize}
\item Content updates get propagated across courses, semesters, and sections,
minimizing the potential for inconsistencies
\item Minimizes redundancy when sharing content across courses, semesters, and
sections
\end{itemize}
\end{itemize}

\begin{itemize}
\item Second, instructors can enjoy the benefits of having a flexible export system and an output-specific export option.
\begin{itemize}
\item Consistent content across multiple output formats
\begin{itemize}
\item Slides (e.g., via reveal.js or Beamer backends)
\item Handouts (e.g., via the \LaTeX{} or reStructuredText backends)
\end{itemize}
\item Selective formatting and presentation of components depending on output
format
\end{itemize}
\item Third, the template contains a) utility functions written in Emacs Lisp, b)
shortcuts to Org mode functions, and c) pre-built tree structures, which
allow automation of many tasks including:
\begin{itemize}
\item Automatic class numbering
\item Automatic creation of key content including (but not limited to)
\begin{itemize}
\item course schedule for syllabi;
\item agenda of lecture materials; and
\item exam keys.
\end{itemize}
\end{itemize}
\end{itemize}
\subsection{Advantages for Students}
\label{sec:org332d9b0}
\begin{itemize}
\item Consistent, properly-formatted, and strategically presented course materials add to student engagement
\item Availability of materials that are easier to digest and review outside the classroom
\end{itemize}
\subsection{Requirements}
\label{sec:org1754caf}
\begin{itemize}
\item Basic knowledge of Emacs
\item Basic knowledge of Org mode
\end{itemize}
\section{Overview of the Directory Structure}
\label{sec:org89a567a}
We present the directory structure of Org-Coursepack.

\begin{description}
\item[{\textbf{/Assets}}] This folder contains:
\begin{itemize}
\item Org setup files, which include frequently used macros (e.g., for LaTex
formatting).
\item Supplementary course materials (if any), such as images, videos, or
articles, for storage and access.
\end{itemize}
\item[{\textbf{/Assets/Institutions}}] This folder contains an institution Org file that
includes institution-specific information (e.g., university policies);
may have multiple Org files if teaching across multiple institutions.

\item[{\textbf{/Courses}}] Each unique course will have a subdirectory under \texttt{Courses}. A
course is defined as a series of lectures occupying a given
adademic calendar unit referred to as a semester. Same courses
may be offered across multiple semesters. Note that a course
may also have multiple sections in the same semester; for
example, a Statistics 101 course may be offered to three
different sets of students per semester.
\item[{\textbf{/Courses/Course}}] This folder contains:

\begin{itemize}
\item A course Org file that includes permanent information about the course
that remains consistent across semesters (e.g., syllabus items such as
learning objectives, grading schemes).
\item A subfolder for each semester this course is taught.
\end{itemize}

\item[{\textbf{/Courses/Course/Semester}}] Each semester folder contains:
\begin{itemize}
\item A semester Org file that includes information about the course that varies
by semester (e.g., classroom location, course schedule, assignment due
dates). The semester Org file also pulls information from other Org files,
such as course, topic, and institution Org files, to complete the course
development for that semester. In other words, this is the master file
that compiles all course materials for exporting.
\item Subfolders are for exported course materials (if any) and are
divided by type; i.e., Assignments, Lectures, Exams, and Syllabus.
\end{itemize}
\item[{\textbf{/Topics}}] This folder contains a topic Org file for each topic; these
files are where course content (e.g., lecture slides and notes,
exam questions, assignment guidelines) about specific topics
are stored and accessed.
\end{description}
\subsection{Example}
\label{sec:org5a1001c}
The following example is the directory structure of this course, Org-Coursepack, as well as the template.

{\footnotesize
\begin{verbatim}
\
|
+---Assets
|   |   setup_Macros.org
|   |
|   +---Institutions
|           JOSE.org
|           Template.org
|
+---Courses
|   +---Org-Coursepack
|   |   |   Org-Coursepack.org
|   |   |
|   |   +---2018 Fall
|   |       |   2018 Fall.org
|   |       |
|   |       +---Assignments
|   |       |   |   Assignment 1.pdf
|   |       |   |   Assignment 1.tex
|   |       |
|   |       +---Lectures
|   |       |   |   01 Introduction.pdf
|   |       |   |   01 Introduction.tex
|   |       |
|   |       +---Exams
|   |       |   |   Exam 1.pdf
|   |       |   |   Exam 1.tex
|   |       |
|   |       +---Syllabus
|               |   Syllabus (Section 1).pdf
|               |   Syllabus (Section 1).tex
|   |
|   +---Template
|       |   Template.org
|       |   
|       +---Semester
|           |   Semester.org
|           |   
|           +---Assignments
|           |   |   Assignment_1.pdf
|           |   |   Assignment_1.tex
|           |           
|           +---Exams
|           +---Lectures
|           |   |   01 Introduction.pdf
|           |   |   01 Introduction.tex
|           |   |   
|           |           
|           +---Syllabus
|               |   Syllabus (Section 1).pdf
|               |   Syllabus (Section 1).tex
|
+---Topics
    |   Org-Teaching.org

\end{verbatim}
}
\end{document}