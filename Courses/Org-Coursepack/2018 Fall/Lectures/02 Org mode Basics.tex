% Intended LaTeX compiler: pdflatex
\documentclass[10pt,article]{article}
\usepackage[utf8]{inputenc}
\usepackage[T1]{fontenc}
\usepackage{graphicx}
\usepackage{grffile}
\usepackage{longtable}
\usepackage{wrapfig}
\usepackage{rotating}
\usepackage[normalem]{ulem}
\usepackage{amsmath}
\usepackage{textcomp}
\usepackage{amssymb}
\usepackage{capt-of}
\usepackage{hyperref}
\usepackage{titling} \posttitle{\par\end{center}} \setlength{\droptitle}{-30pt} \usepackage{multicol} \setlength{\columnsep}{1cm} \usepackage[T1]{fontenc} \usepackage[utf8]{inputenc} \renewcommand{\contentsname}{Table of Contents / Agenda} \usepackage[letterpaper,left=1in,right=1in,top=0.7in,bottom=1in,headheight=23pt,includehead,includefoot,heightrounded]{geometry} \usepackage{fancyhdr} \pagestyle{fancy} \fancyhf{} \cfoot{\thepage} \usepackage{mathpazo} \usepackage[scaled=0.85]{helvet} \usepackage{courier} \usepackage[onehalfspacing]{setspace} \usepackage[framemethod=default]{mdframed} \usepackage{wrapfig} \usepackage{booktabs} \usepackage[outputdir=Lectures]{minted}
\setcounter{secnumdepth}{3}
\date{\vspace{-6ex}}
\title{Class 2: Org Mode Basics}
\hypersetup{
 pdfauthor={},
 pdftitle={Class 2: Org Mode Basics},
 pdfkeywords={},
 pdfsubject={Org-Coursepack-Documentation: A Comprehensive Course Template for Teaching with Org Mode School specific teaching materials},
 pdfcreator={Emacs 25.3.1 (Org mode 9.1.13)}, 
 pdflang={English}}
\begin{document}

\maketitle
\lhead{ ORG 0000 \\ Joon H. Ro \& Jae-Eun Namkoong } 
\rhead{ Class 2 \\ 2018-08-30 Thu} 
\thispagestyle{fancy}

\setcounter{tocdepth}{1}
\tableofcontents
\vspace{6ex}

\section{Headings}
\label{sec:orgd916d22}
\begin{itemize}
\item In Org mode, a heading is denoted by preceding stars (\texttt{*}), and the level of
a heading is determined by the number of stars.
\item For example,

\begin{verbatim}
  * Heading 1
  ** Subheading 1.1
  * Heading 2
  ** Subheading 2.1
\end{verbatim}

\item A heading can have tags and properties associated with them, which allows
very flexible usage that we exploit throughout Org-Coursepack.
\end{itemize}
\subsection{Tags}
\label{sec:orgd79f8e8}
\begin{itemize}
\item A heading can have tags associated with them. They are used in the form of \texttt{:tag:}, on the same line as the heading. For example,

\begin{verbatim}
  * Heading with a tag                       :tag:
\end{verbatim}
\end{itemize}
\subsection{Properties}
\label{sec:orgacf5440}
\begin{itemize}
\item Each heading can have properties associated with it. You can think of
properties as data specific for the heading, as they consist of
\texttt{:PROPERTY\_NAME:} and \texttt{PROPERTY VALUE} pairs.
\item For example, a heading for a lecture can have property values for lecture
number, file name for exports, etc. One of the frequently used properties is
\texttt{:CUSTOM\_ID:}, which can be used to locate the particular heading. See the
\texttt{Including Content from Another Org file} section for how it is used in
Org-Coursepack.
\item See the \href{https://orgmode.org/manual/Properties-and-columns.html}{Org manual} for more information.

\item For example,
\begin{verbatim}
  * Heading 1
  :PROPERTIES:
  :CUSTOM_ID: Heading 1
  :END:
\end{verbatim}
\end{itemize}
\section{Macros}
\label{sec:orge1f0fd8}
Org mode provides macro replacement functionality. This can be especially
useful for instructors, as shown in the following cases.

\begin{itemize}
\item When defining terms that will be repeated within or across course-related
materials, or needs to be changed in future semesters.
\begin{description}
\item[{Example}] Suppose an instructor were to present a due date for
"Assignment 1" in three different places (e.g., syllabus,
lecture, and assignment guideline). By using a macro, an
instructor can ensure that all documents have the same
information, and make such updates easily in future
semesters. Such a macro (named \texttt{DUE\_ASSIGNMENT\_1}) can be
defined like the following:

\begin{verbatim}
    #+MACRO: DUE_ASSIGNMENT_1 [2018-09-27 Thu]
\end{verbatim}

One can then use the macro with \texttt{\{\{\{MACRO\_NAME\}\}\}}. For example, all
areas in documents where it says \texttt{\{\{\{DUE\_ASSIGNMENT\_1\}\}\}} will be
replaced with \texttt{[2018-09-27 Thu]} upon export.
\end{description}

\item When defining terms specific to a subtree (e.g., a specific class) that
the instructor may want to use in multiple places. This can be useful when
specifying class orders, class titles, or section numbers. One
can achieve this by referring to property values of a subtree.
\begin{description}
\item[{Example}] An instructor can define a class number property like the following:

\begin{verbatim}
    * Heading
    :PROPERTIES:
    :CLASS:    2
    :END:
\end{verbatim}

Then \texttt{\{\{\{property(PROPERTY\_NAME)\}\}\}} will be replaced by
the value of \texttt{PROPERTY\_NAME} property:

\begin{verbatim}
    This is class number {{{property(CLASS)}}}.
\end{verbatim}
\end{description}

\item See the \href{https://orgmode.org/manual/Macro-replacement.html}{Org manual} for more information.
\end{itemize}
\section{Including Content from Another Org file}
\label{sec:orgb5a741c}
\begin{itemize}
\item In Org mode, a user can pull content, without making a copy, from any org file (including
the current one) via the \texttt{\#+INCLUDE:} statement. See the \href{https://orgmode.org/manual/Include-files.html}{Org manual} fore more information.

\item Taking advantage of this functionality, the template is created so that its
content is modular and can be pulled flexibly as needed. This feature is
useful when sharing content across different courses or semesters or when
revisiting past materials (e.g., in exam reviews or when reviewing past
cases to build on them).

\begin{description}
\item[{Example}] This is how an instructor would use the \texttt{\#+INCLUDE:} statement
to pull content from a subtree with \texttt{:CUSTOM\_ID:} \texttt{R-squared}
in the file \texttt{Regression.org} so it is presented across multiple
courses (i.e., Statistics 101 and Marketing Research).
\texttt{:only-contents t} option means only the contents of the
subtree, not its heading and properties, will be included. To
visit the included file, press \texttt{C-c '} while the cursor is on
the \texttt{\#+INCLUDE:} statement.

\begin{itemize}
\item \textbf{\texttt{Regression.org}} \quad 

{\small
\begin{verbatim}
      * R-squared
      :PROPERTIES:
      :CUSTOM_ID: R-squared
      :END:
      - The definition of \( R^{2} \) is:
        \[ R^{2} = 1 - \dfrac{SS_{\text{res}}}{SS_{\text{tot}}}\]
\end{verbatim}

\item \textbf{\texttt{Statistics 101.org}} \quad 

\begin{verbatim}
      * Regression
      ** R-squared
      #+INCLUDE: "/Regression.org::#R-squared" :only-contents t
\end{verbatim}

\item \textbf{\texttt{Marketing Research.org}} \quad 

\begin{verbatim}
      * Regression
      ** R-squared
      #+INCLUDE: "/Regression.org::#R-squared" :only-contents t
\end{verbatim}
}
\end{itemize}
\end{description}
\end{itemize}
\end{document}