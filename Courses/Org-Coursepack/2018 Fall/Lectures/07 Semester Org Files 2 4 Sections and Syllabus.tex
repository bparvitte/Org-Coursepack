% Intended LaTeX compiler: pdflatex
\documentclass[10pt,article]{article}
\usepackage[utf8]{inputenc}
\usepackage[T1]{fontenc}
\usepackage{graphicx}
\usepackage{grffile}
\usepackage{longtable}
\usepackage{wrapfig}
\usepackage{rotating}
\usepackage[normalem]{ulem}
\usepackage{amsmath}
\usepackage{textcomp}
\usepackage{amssymb}
\usepackage{capt-of}
\usepackage{hyperref}
\usepackage{titling} \posttitle{\par\end{center}} \setlength{\droptitle}{-30pt} \usepackage{multicol} \setlength{\columnsep}{1cm} \usepackage[T1]{fontenc} \usepackage[utf8]{inputenc} \renewcommand{\contentsname}{Table of Contents / Agenda} \usepackage[letterpaper,left=1in,right=1in,top=0.7in,bottom=1in,headheight=23pt,includehead,includefoot,heightrounded]{geometry} \usepackage{fancyhdr} \pagestyle{fancy} \fancyhf{} \cfoot{\thepage} \usepackage{mathpazo} \usepackage[scaled=0.85]{helvet} \usepackage{courier} \usepackage[onehalfspacing]{setspace} \usepackage[framemethod=default]{mdframed} \usepackage{wrapfig} \usepackage{booktabs} \usepackage[outputdir=Classes]{minted}
\setcounter{secnumdepth}{3}
\date{\vspace{-6ex}}
\title{Class 7: Semester Org Files (2/4): Sections and Syllabus}
\hypersetup{
 pdfauthor={},
 pdftitle={Class 7: Semester Org Files (2/4): Sections and Syllabus},
 pdfkeywords={},
 pdfsubject={Org-Coursepack-Documentation: A Comprehensive Course Template for Teaching with Org Mode School specific teaching materials},
 pdfcreator={Emacs 25.3.1 (Org mode 9.1.13)}, 
 pdflang={English}}
\begin{document}

\maketitle
\lhead{ ORG 0000 \\ Joon H. Ro \& Jae-Eun Namkoong } 
\rhead{ Class 7 \\ 2018-09-18 Tue} 
\thispagestyle{fancy}

\setcounter{tocdepth}{1}
\tableofcontents
\vspace{6ex}

\section{Sections Subtree}
\label{sec:orgb518f6b}
An instructor may teach multiple sections of the same course during a
given semester. The \texttt{Sections} subtree contains a subtree for each
section, which contains section-specific information such as section
number, classroom location and time, and exam dates. It also acts as a
wrapper around the content of the syllabus, which passes on section-specific
information.

\begin{verbatim}
  * Sections
  :PROPERTIES:
  ** 01
  ** 02
  ** 03
\end{verbatim}
\subsection{\texttt{:PROPERTIES:}}
\label{sec:org1f2d8ed}
\begin{itemize}
\item :Properties: of these subtrees have information common across all sections,
such as \LaTeX{} preamble items via \texttt{EXPORT\_LATEX\_HEADER}. These will be
inherited and shared by the child subtrees with the
\texttt{org-use-property-inheritance} option set to \texttt{t}. For example:
{\footnotesize
\begin{minted}[]{latex}
  * Sections
  :PROPERTIES:
  :CUSTOM_ID: Sections
  :EXPORT_LATEX_CLASS: koma-article
  :EXPORT_LATEX_CLASS_OPTIONS: [article,letterpaper,times,10pt,listings-bw,microtype]
  :EXPORT_LATEX_HEADER+: \usepackage[onehalfspacing]{setspace}
  :EXPORT_LATEX_HEADER+: \usepackage[T1]{fontenc}
  :EXPORT_LATEX_HEADER+: \usepackage{mathpazo} \usepackage[scaled=0.85]{helvet} \usepackage{courier}
  :EXPORT_LATEX_HEADER+: \usepackage{geometry} \geometry{left=1in,right=1in,top=1in,bottom=1in}
  :EXPORT_LATEX_HEADER+: \usepackage[framemethod=default]{mdframed}
  :EXPORT_DATE: {{{SEMESTER}}}
  :EXPORT_OPTIONS: num:nil title:nil toc:nil tags:nil
  :END:
\end{minted}
}
\end{itemize}
\subsection{Sections}
\label{sec:org20b9d5a}
\begin{itemize}
\item Each section subtree has properties and macros for the section
information and a child subtree for the syllabus of the section:

\begin{verbatim}
  ** 01
  :PROPERTIES:
  :MACROS_Section_Info:
  *** Syllabus
\end{verbatim}
\end{itemize}
\subsubsection{\texttt{:PROPERTIES:}}
\label{sec:org8396f39}
Each section subtree has property items containing section-specific
information such as \texttt{:SECTION:} (section number), \texttt{:SECTION\_LOC:}
(classroom location), \texttt{:SECTION\_DATE:} (class date and time), and
\texttt{:SECTION\_DATE\_FINAL\_EXAM:} (final exam date), which will be used in
the syllabus via the \texttt{\{\{\{property(Property\_NAME)\}\}\}} grammar (e.g.,
\texttt{\{\{\{property(SECTION\_DATE)\}\}\}}). Note that we created a macro
\texttt{\{\{\{DATE\_FINAL\_EXAM\_01\}\}\}} for \texttt{:SECTION\_DATE\_FINAL\_EXAM:}, so it can be
used in other places, such as class announcements.

\begin{verbatim}
  ** 01
  :PROPERTIES:
  :SECTION: 01
  :SECTION_DATE: Tue/Thurs, 9:30a-10:45
  :SECTION_DATE_FINAL_EXAM: {{{DATE_FINAL_EXAM_01}}}
  :SECTION_LOC: BLDG 100
  :CUSTOM_ID: Sections/01
  :END:
\end{verbatim}
\subsubsection{Macros}
\label{sec:org33022bc}
\begin{itemize}
\item Instructors can specify the final exam date in the
\texttt{:MACROS\_Section\_Info:} drawer.
\item To allow this information to be used in other places (e.g.,
reminders or announcements), we used a macro instead of a property.

\begin{verbatim}
  ** 01
  :PROPERTIES:
  :MACROS_Section_Info:
  #+MACRO: DATE_FINAL_EXAM_01 [2018-12-16 Sun 13:00] - 4:00PM
  :END:
\end{verbatim}
\end{itemize}
\subsubsection{Syllabus}
\label{sec:orgb0c369a}
\begin{itemize}
\item The \texttt{Syllabus} child subtree has properties containing section-specific
information for the section's syllabus. For its body, it pulls content
from the \texttt{Syllabus} subtree of the file.
\begin{verbatim}
  *** Syllabus
  :PROPERTIES:
  #+INCLUDE: "./2018 Fall.org::#Syllabus" :only-contents t
\end{verbatim}
\item \texttt{:PROPERTIES:} of this subtree contain export-related information for the
syllabus, such as \texttt{:EXPORT\_FILE\_NAME:}.

\begin{verbatim}
  *** Syllabus
  :PROPERTIES:
  :EXPORT_TITLE: {{{COURSE_NUM}}}-{{{property(SECTION)}}} Syllabus
  :EXPORT_FILE_NAME: ./Syllabus/01_Syllabus
  :EXPORT_TO:  LaTeX (Custom Time Format)
  :OUTPUT_VIEW: PDF
  :END:
\end{verbatim}
\item The syllabus subtree simply includes in its body the content of the top level \texttt{Syllabus} tree. Since
property macros are used for section-specific information in the content,
the corresponding section-specific information will be included automatically.

\begin{verbatim}
  *** Syllabus
  :PROPERTIES:
  #+INCLUDE: "./2018 Fall.org::#Syllabus" :only-contents t
\end{verbatim}
\end{itemize}
\section{Syllabus Subtree}
\label{sec:org2aad40e}
The \texttt{Syllabus} subtree contains course description, learning
objectives, grading, etc., as shown in the example below. Instructors
can include additional information as they see fit. We now describe each
child subtree.

\begin{verbatim}
  * Syllabus
  ** Tasks [0/1]                                                     :noexport:
  ** Intro                                                            :ignore:
  ** Course Description
  ** Course Prerequisites
  ** Student Learning Objectives
  ** Course Material
  ** Grading
  ** Specific Course Policies
  ** School-Specific Policies
  ** Class Schedule                                                   :newpage:
\end{verbatim}
\subsection{Intro}
\label{sec:orgfcc4742}
The Intro has the title page of the Syllabus. The title page contains a blank
school name (the color of the text is defined in the institution Org file) as a
placeholder for the school logo the instructor wishes to use. There is
also a table of course information, which heavily relies on macros
to avoid redundancy.

Some of the macros are in the form of \texttt{property(PROPERTY\_NAME)} (e.g.,
\texttt{\{\{\{property(SECTION\_DATE)\}\}\}}), which means it will receive the value of the
\texttt{:PROPERTY\_NAME:} property of the subtree. When the content is included in
another subtree such as the \texttt{Sections} subtree, these property macros will pull
the value from the subtree including this content. That is,
\texttt{\{\{\{property(SECTION\_DATE)\}\}\}} will get the value from the \texttt{:SECTION\_DATE:}
property of the \texttt{01} (\texttt{02}) child subtree of the \texttt{Sections} subtree when
included by the \texttt{01} (\texttt{02}) subtree.

Note that the header of the Intro subtree is assigned an \texttt{:ignore:} tag, which means the heading (\texttt{Intro})
will be ignored in exporting.

{\scriptsize
\begin{verbatim}
  ** Intro                                                            :ignore:
  @@latex:\definecolor{SchoolColor}{RGB}{@@{{{SCHOOL_COLOR}}}@@latex:}@@

  #+BEGIN_CENTER
  #+LATEX: {\color{SchoolColor}{\Large
  *{{{SCHOOL}}}*
  #+LATEX: }}
  #+END_CENTER
  
  #+BEGIN_CENTER
  #+LATEX: {\color{SchoolColor}{
  *{{{COURSE_NUM}}}-{{{property(SECTION)}}}*
  
  *{{{COURSE}}}*
  
  *{{{SEMESTER}}}*
  #+LATEX: }}
  #+END_CENTER
  
  {{{VSPACE(5)}}}
  
  | *Instructor:*               | {{{PROFESSOR}}}              | *Office Phone:*   | {{{PHONE}}}                 |
  | *Office:*                   | {{{OFFICE}}}                 | *E-mail:*         | {{{EMAIL}}}                 |
  | *Office Hours:*             | {{{OFFICE_HOURS}}}           | *Course Site:*    | *{{{COURSE_LINK}}}*         |
  | *Class Meeting Day & Time:* | {{{property(SECTION_DATE)}}} | *Class Location:* | {{{property(SECTION_LOC)}}} |
\end{verbatim}
}
\subsection{Course-speific information}
\label{sec:org200c51f}
\begin{itemize}
\item Since different sections of the same course typically share common elements (e.g.,
course description, prerequisites, learning objectives), the
content of most child subtrees of \texttt{Syllabus} are pulled from the course Org
file on exporting, using \texttt{\#+INCLUDE}.
\item Note that instructors can use semester-specific information in these child
subtrees using macros -- see the \texttt{Grading} child subtree for an example.
\item School-wide information (e.g., grade cutoff percentages) is imported
from the institution Org file.
\end{itemize}
\subsection{Class Schedule}
\label{sec:org810f02d}
\begin{itemize}
\item The \texttt{Class Schedule} subtree contains the class schedule in a table format.
\item The table is dynamically generated using org-mode's \href{https://orgmode.org/manual/Capturing-column-view.html\#Capturing-column-view}{columnview dynamic block}
functionality -- it will extract information about each class from the class
subtree's properties, and automatically create the class schedule table.

\item One can use \texttt{C-c C-c} while the cursor is on the \texttt{\#+BEGIN: columnview} to
update the columnview dynamic block. Once it is updated, \texttt{\#+TBLFM} is used
to format the table (e.g., change the third column name from \texttt{ITEM} to
\texttt{Topic}) automatically.
\end{itemize}
\end{document}