% Intended LaTeX compiler: pdflatex
\documentclass[10pt,article]{article}
\usepackage[utf8]{inputenc}
\usepackage[T1]{fontenc}
\usepackage{graphicx}
\usepackage{grffile}
\usepackage{longtable}
\usepackage{wrapfig}
\usepackage{rotating}
\usepackage[normalem]{ulem}
\usepackage{amsmath}
\usepackage{textcomp}
\usepackage{amssymb}
\usepackage{capt-of}
\usepackage{hyperref}
\usepackage{titling} \posttitle{\par\end{center}} \setlength{\droptitle}{-30pt} \usepackage{multicol} \setlength{\columnsep}{1cm} \usepackage[T1]{fontenc} \usepackage[utf8]{inputenc} \renewcommand{\contentsname}{Table of Contents / Agenda} \usepackage[letterpaper,left=1in,right=1in,top=0.7in,bottom=1in,headheight=23pt,includehead,includefoot,heightrounded]{geometry} \usepackage{fancyhdr} \pagestyle{fancy} \fancyhf{} \cfoot{\thepage} \usepackage{mathpazo} \usepackage[scaled=0.85]{helvet} \usepackage{courier} \usepackage[onehalfspacing]{setspace} \usepackage[framemethod=default]{mdframed} \usepackage{wrapfig} \usepackage{booktabs} \usepackage[outputdir=Classes]{minted}
\setcounter{secnumdepth}{3}
\date{\vspace{-6ex}}
\title{Class 8: Semester Org Files (3/4): Lectures}
\hypersetup{
 pdfauthor={},
 pdftitle={Class 8: Semester Org Files (3/4): Lectures},
 pdfkeywords={},
 pdfsubject={Org-Coursepack-Documentation: A Comprehensive Course Template for Teaching with Org Mode School specific teaching materials},
 pdfcreator={Emacs 25.3.1 (Org mode 9.1.13)}, 
 pdflang={English}}
\begin{document}

\maketitle
\lhead{ ORG 0000 \\ Joon H. Ro \& Jae-Eun Namkoong } 
\rhead{ Class 8 \\ 2018-09-20 Thu} 
\thispagestyle{fancy}

\setcounter{tocdepth}{1}
\tableofcontents
\vspace{6ex}

\section{The \texttt{Lectures} Top-level Tree}
\label{sec:orgdb3c315}
\begin{itemize}
\item The \texttt{Lectures} top-level tree, which is the basis for the class
schedule table, contains a subtree for each class and event (e.g.,
assignment, exam). See the example below.

{\small
\begin{verbatim}
  * Lectures
  :PROPERTIES:
  
  #+NAME: Update Classes
  #+BEGIN_SRC emacs-lisp :results none
  
  ** Tasks [0/1]                                           :noexport:skipcount:
  ** Classes and Assignments Dates                                  :skipcount:
  ** Introduction
  ** Org Mode Basics
  ** ...
  ** _Assignment 1 Due_                     :Assignment:skipcount:
  ** _Exam 1_                                                           :Exam:
  ** *Thanksgiving Holiday*                                 :Holiday:skipcount:
  ** _Final Exam_                                                   :skipcount:
\end{verbatim}
}
\end{itemize}
\subsection{\texttt{:PROPERTIES:}}
\label{sec:orgb2921a9}
\begin{itemize}
\item The properties of the \texttt{Lectures} top-level tree contain information common
across lectures such as export options, including the handout \LaTeX{}
headers (e.g., syntax highlighting options for the \texttt{minted} package). With
the \texttt{org-use-property-inheritance} option set to \texttt{t}, the settings will be
propagated to its subtrees. For example, the \texttt{:EXPORT\_LATEX\_HEADER+:}
property items, which specify the preamble for \LaTeX{} lecture handouts, will
be shared by all lecture subtrees.

\item The \texttt{COLUMNS} and \texttt{ID} properties of the \texttt{Lectures} top-level tree
are used to create the class schedule columnview dynamic block
described in \texttt{Syllabus}.

{\footnotesize
\begin{verbatim}
  :PROPERTIES:
  :COLUMNS: %Date %Class %ITEM
  :ID:  79d5e887-4637-43e7-8e8a-b83fa83ee56e
  ...
  :END:
\end{verbatim}
}
\end{itemize}

\subsection{\texttt{Update Lectures} Source Code Block}
\label{sec:org7b94c50}
\begin{itemize}
\item The \texttt{Lectures} top-level tree has a source code block named \texttt{Update
  Lectures}. When executed with \texttt{C-c C-c}, it is designed to go through
each lecture subtree and perform the following actions:

\begin{enumerate}
\item Update \texttt{:PROPERTIES:} for the lecture, such as the class number
(\texttt{:CLASS:}), class date (\texttt{:DATE:}), and file name of the export
(\texttt{:EXPORT\_FILE\_NAME:}).

\begin{description}
\item[{\texttt{:CLASS:}}] The lectures are assumed to be in the order of the
lecture schedule (e.g., first lecture on Class
1). Note that any subtree with a \texttt{:skipcount:} tag
will be ignored, which is useful for
non-lecture subtrees (e.g., assignment due dates
and holidays).
\item[{\texttt{:DATE:}}] It will get the date of each class from the \texttt{DATE\_CLASS\_XX} file-level
properties, which are defined in the \texttt{Lectures and
                   Assignments Dates} subtree.
\item[{\texttt{:EXPORT\_FILE\_NAME:}}] By default, the script sets the
\texttt{:EXPORT\_FILE\_NAME:} as the subtree heading, which can be
overridden by setting the \texttt{:EXPORT\_FILE\_NAME\_MANUAL:}
property of the lecture subtree. If the property exists, the
script will use its value for \texttt{:EXPORT\_FILE\_NAME:}
instead. This is useful when the lecture subtree heading is
very long or contains invalid characters for a file name.
\end{description}

\item Update \texttt{Lecture Agenda} under the \texttt{Introduction} subtree.

\begin{description}
\item[{\texttt{Lecture Agenda}}] The script will get the list of subtrees
that belong to the particular lecture, ignoring any with
\texttt{noexport}, \texttt{handoutonly}, or \texttt{slideonly} tags. Then, it
will insert the list into the body of \texttt{Lecture Agenda}. In
addition, it will set the \texttt{CUSTOM\_ID} property value of the
subtree accordingly, so the agenda can be used in other
places.
\end{description}

\item Update \texttt{Last Class} under the \texttt{Introduction} subtree and \texttt{Class Summary} of
each lecture. 

\begin{description}
\item[{\texttt{Last Class}}] The script will insert an \texttt{\#+INCLUDE:}
statement which points to the previous
lecture's \texttt{Lecture Agenda} subtree under the
\texttt{Introduction}. This is to provide a recap of
the previous lecture prior to starting the
current lecture.

\item[{\texttt{Class Summary}}] The script will insert an \texttt{\#+INCLUDE:}
statement which points to the current lecture's \texttt{Lecture
          Agenda} subtree under the \texttt{Introduction}. This provides a
summary of the current lecture.
\end{description}
\end{enumerate}

\item The user should run this script before updating the class schedule table 
in the \texttt{Syllabus}, so the most current information is reflected in the table.
\end{itemize}
\subsection{\texttt{Lectures and Assignments Dates} Subtree}
\label{sec:org54ba724}
\begin{itemize}
\item In this subtree, instructors can define lecture dates and assignment
due dates as file-level properties. For example,

\begin{verbatim}
  #+MACRO: DUE_ASSIGNMENT_1 [2018-09-27 Thu]
  #+MACRO: DUE_ASSIGNMENT_2 [2018-10-30 Tue]
  
  #+DATE_CLASS_01: [2018-08-28 Tue]
  #+DATE_CLASS_02: [2018-08-30 Thu]
\end{verbatim}

\item The \texttt{Update Lectures} source code block will use the dates defined
in the file-level properties as shown above to update the date of each lecture.

\item 28 lecture/class dates are pre-defined in the
Org-Coursepack. Instrutors can easily customize them to meet their needs.
\end{itemize}
\subsection{\texttt{Common Items} Subtree}
\label{sec:org2f2a6c4}
\begin{itemize}
\item The \texttt{Common Items} subtree has common items across all lectures. Currently
there is one subtree, \texttt{Handout heading}, which contains \LaTeX{} codes for
header items and table of contents. The content will be included from the 
\texttt{Handout heading} subtree of each individual lecture subtree.
\end{itemize}
\subsection{Dynamic Columnview of Lectures}
\label{sec:org8adcc75}
\begin{itemize}
\item A useful functionality of Org mode is the ability to create a
table-view overlay of subtrees with their property
values. Instructors can use the \texttt{org-columns} command to create a column-view of
lectures, which is essentially the same as the class schedule table
in the \texttt{Syllabus}. It is useful when there is a need to quickly inspect
the overall course schedule.
\end{itemize}
\section{Individual Lecture Subtree}
\label{sec:org04f79ed}
Each lecture subtree contains the teaching materials for that particular lecture/class. The
example below shows the general structure of the subtree.

\begin{verbatim}
  ** Introduction
  :PROPERTIES:
  *** Tasks [0/1]                                                  :noexport:
  *** Handout heading                                    :handoutonly:ignore:
  *** Introduction                                                :slideonly:
  *** Introduction to {{{COURSE}}}
  *** Overview of the Directory Structure
  *** Summary                                                     :slideonly:
\end{verbatim}
\subsection{\texttt{:PROPERTIES:}}
\label{sec:orgfffbf61}
\begin{itemize}
\item A lecture subtree has properties containing lecture-specific information.
\item As described earlier, \texttt{:CLASS:} (class number), \texttt{:EXPORT\_FILE\_NAME:}, and \texttt{:DATE:} (class date) will be
automatically updated by the \texttt{Update Lectures} source code block.
\item The \texttt{:EXPORT\_TO:} property has clickable links written in Emacs-lisp, which will
export class content to the designated output format. For example, clicking
\texttt{reveal.js} will export content to reveal.js slides.
\item The \texttt{:OUTPUT\_VIEW:} property has links that, when clicked,
opens the corresponding output files, such as html or pdf files. The
links will use the value of the \texttt{:EXPORT\_FILE\_NAME:} property as the
file path; hence, it is unnecessary to manually edit the output
links.

{\small
\begin{verbatim}
  ** Introduction
  :PROPERTIES:
  :CLASS:    1
  :EXPORT_TITLE: Class {{{property(CLASS)}}}: {{{property(ITEM)}}}
  :EXPORT_FILE_NAME: ./Classes/01 Introduction
  :DATE:     [2018-08-28 Tue]
  :EXPORT_TO:  reveal.js | Beamer | LaTeX 
  :OUTPUT_VIEW: HTML | PDF
  :END:
\end{verbatim}
}
\end{itemize}
\subsection{Tasks}
\label{sec:org94f68b4}
The \texttt{Tasks} subtree contains lecture-specific tasks you may have as an
instructor. These are presented in the form of \href{https://orgmode.org/manual/TODO-items.html}{Org mode TODO
items}. The \texttt{:noexport:} tag prevents the tree from being exported.
\subsection{Handout heading}
\label{sec:orge949136}
The \texttt{Handout heading} subtree will only be included in a handout export (with the
\texttt{:handoutonly:} tag). It includes the content from \texttt{Handout heading} subtree of the 
\texttt{Common Items} subtree in the \texttt{Lectures} top-level tree.
\subsection{Introduction}
\label{sec:org6f5f708}
The \texttt{Introduction} has three subheadings:

{\small
\begin{verbatim}
  *** Introduction                                                :slideonly:
  **** Announcements
  **** Last Class
  **** Lecture Agenda
\end{verbatim}
}

Instructors can enter any announcements to be made in class in \texttt{Annoucements};
\texttt{Last Class} includes a recap of the learning objectives from the previous
class; \texttt{Lecture Agenda} lists the learning objectives for the current lecture.

Note that the content (\texttt{\#+INCLUDE:} statements) and properties (e.g.,
\texttt{CUSTOM\_ID}) of the latter two subheadings will be automatically updated by the \texttt{Update Lectures}
script as discussed earlier.

With the \texttt{:slideonly:} tag, the \texttt{Introduction} will only be exported to slide
outputs.
\subsection{Content}
\label{sec:org9ac3224}
Subtrees following the \texttt{Introduction} subtree contain lecture content.
To minimize redunancy, lectures should draw as much material from the reusable
content in the topic Org file subtrees as possible. See example
below. (For more examples, see \texttt{2018 Fall.org} in \texttt{Org\_Teaching}.)

{\footnotesize
\begin{verbatim}
  *** Topic Org Files
  #+INCLUDE: "../../../Topics/Org_Teaching.org::#Lectures/Topic Org Files" :only-contents t
  *** Course Org Files
  #+INCLUDE: "../../../Topics/Org_Teaching.org::#Lectures/Course Org Files" :only-contents t
\end{verbatim}
}

\subsection{Class Summary}
\label{sec:orgd89805c}
The \texttt{Class Summary} reviews the content of the current class, by including the content of \texttt{Lecture
Agenda} in the \texttt{Introduction} subtree.  The \texttt{\#+INCLUDE:} statement will be
automatically generated by the \texttt{Update Lectures} script as described earlier.
With the \texttt{:slideonly:} tag, \texttt{Class Summary} will only be exported in slide outputs.
\end{document}